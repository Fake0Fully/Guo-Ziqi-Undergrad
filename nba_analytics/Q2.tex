\documentclass[12pt]{article}
\pagestyle{empty}
\newcommand\tab[1][1cm]{\hspace*{#1}}
\newcommand{\argmin}{\operatornamewithlimits{argmin}}
\newcommand{\argmax}{\operatornamewithlimits{argmax}}
\usepackage[margin=1in]{geometry}
\usepackage{array}
\usepackage{amsmath}
%\usepackage{cyrillic}
\usepackage{graphicx}
\usepackage{subcaption}
\usepackage{float}

\thispagestyle{empty}

\begin{document}
\begin{center}
\large\bf NBA Playoff Contention: Solution Concept
\end{center}
\begin{center}
\large
Harvard University - Ziqi Guo
\end{center}
\begin{itemize}
\item The \texttt{league.py} file provides the class definitions of \texttt{match} and \texttt{league}. 
\medskip\\In the class \texttt{league}, various functions are defined which can be used to calculate relevant statistics, such as the win percentage in division and conference, etc. These functions are used to define a \texttt{tie\_breaker()} function that determines the winner of two teams based on the tie-breaking procedures provided.
\item The \texttt{elimination.py} file imports data, instantiate the \texttt{league} class and obtains the elimination dates.
\item The way to check for elimination is by considering the best possible records for the remainder of the games. If a team still could not achieve a higher win percentage than the $8^{th}$ place team under the best-case scenario, then we consider it eliminated for playoff contention. The procedure is as follows: 
\begin{enumerate}
\item{} When a game is played, check every team (call it team A) and compare its records with the team that is currently at $8^{th}$ place (call it team B) in the same conference. If team A could not have more wins than team B, given that team A wins all its remaining teams and team B loses all its remaining teams, then we consider team A dangerous.
\item{} Simulate the rest of the games such that team A wins all and team b loses all.
\item{} Apply the tie-breaking procedure defined in \texttt{league.py}. If team A loses, then it is eliminated.
\end{enumerate}
\item The current tie-breaking procedures only compare the team of interest and the team at $8^{th}$ of the conference. But in fact, they could form tie with the $7^{th}$ place, or even more. Therefore, the tie-breaking procedure can be extended in a similar way to accommodate more than 2 teams at tie.
\end{itemize}

\end{document}